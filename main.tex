\documentclass[12pt]{article}

% Encoding and language
\usepackage[utf8]{inputenc}
\usepackage[T1]{fontenc}
\usepackage[english]{babel}

% Page layout
\usepackage[margin=1in]{geometry}

% Math and symbols
\usepackage{amsmath,amssymb,amsfonts}

% For quotes & punctuation with biblatex + babel
\usepackage{csquotes}

\usepackage[
  backend=biber,
  style=numeric,
  sorting=none,
  giveninits=true,
  doi=true,
  url=false,
  isbn=false,
  eprint=false,
  maxbibnames=10,    % Elsevier typically uses 10 authors before et al.
  minbibnames=10,    % Ensures at least 10 authors appear before et al.
  maxcitenames=2,
  uniquename=false
]{biblatex}
\addbibresource{references.bib}

% Author initials first (Z. Gu)
\DeclareNameAlias{author}{given-family}

% No "In:" prefix
\renewbibmacro{in:}{}

% Title without quotation marks or italics
\DeclareFieldFormat[article,inproceedings,incollection]{title}{#1}

% Pages format without prefix
\DeclareFieldFormat[article]{pages}{#1}

% DOI without prefix
\DeclareFieldFormat{doi}{#1}

% Hyperlinks
\usepackage{hyperref}
\hypersetup{
  colorlinks  = true,
  linkcolor   = blue,
  citecolor   = blue,
  urlcolor    = magenta
}
\usepackage{url}

% Graphics, tables, and floats
\usepackage{graphicx}
\usepackage{booktabs}
\usepackage{xltabular}
\usepackage{float}

% For referencing figures/tables with bold labels
\usepackage[labelfont=bf]{caption}

% Optional line spacing
\usepackage{setspace}
%\onehalfspacing

% -------------------------------
% Code Listings Configuration
% -------------------------------
\usepackage{listings}
\usepackage{xcolor} % for custom colors

% Define a custom style for Python code
\lstdefinestyle{myCustomPython}{
  language=Python,
  basicstyle=\ttfamily\footnotesize,
  keywordstyle=\color{blue}\bfseries,
  commentstyle=\color{green!50!black}\itshape,
  stringstyle=\color{red},
  backgroundcolor=\color{gray!10},
  frame=single,
  rulecolor=\color{gray},
  numbers=left,
  numberstyle=\tiny\color{gray},
  stepnumber=1,
  breaklines=true,
  breakatwhitespace=true,
  showstringspaces=false,
  captionpos=b,
  tabsize=4,
  columns=fullflexible
}
\lstset{style=myCustomPython}
% -------------------------------

% For rotating (landscape) tables
\usepackage{rotating}

%%%%%%%%%%%%%%%%%%%%%%%%%%%%%%%%%%%%%%%%%%%%%%%%%%%%%%%%%%%%
%              BEGIN DOCUMENT
%%%%%%%%%%%%%%%%%%%%%%%%%%%%%%%%%%%%%%%%%%%%%%%%%%%%%%%%%%%%
\begin{document}

\title{Integrating Multi-Index Approaches and Real-Time Data for Enhanced Irrigation Scheduling: A Comprehensive Review and Case Study}
\author{%
Bryan Nsoh\textsuperscript{1,2},
Abia Katimbo\textsuperscript{1,3,*},
Kendall C. DeJonge\textsuperscript{4},
Daran R. Rudnick\textsuperscript{5},
Weizhen Liang\textsuperscript{6},
Derek M. Heeren\textsuperscript{1},
Yeyin Shi\textsuperscript{1},
Hongzhi Guo\textsuperscript{7},
Yufeng Ge\textsuperscript{1},
Xin Qiao\textsuperscript{6},
Birru Girma\textsuperscript{8},
Hope Njuki Nakabuye\textsuperscript{9},
Issa Kabenge\textsuperscript{10},
Joshua Wanyama\textsuperscript{10}
}
\date{\vspace{-1em}} % optional
\maketitle

\noindent
\textsuperscript{1}Department of Biological Systems Engineering, University of Nebraska--Lincoln, Lincoln, NE 68583, USA\\
\textsuperscript{2}West Central Research, Extension, and Education Center, University of Nebraska--Lincoln, North Platte, NE 69101, USA\\
\textsuperscript{3}Department of Electrical and Computer Engineering, University of Nebraska--Lincoln, Lincoln, NE 68588, USA\\
\textsuperscript{4}Water Management and Systems Research Unit, USDA Agricultural Research Service, Fort Collins, CO 80526, USA\\
\textsuperscript{5}Carl and Melinda Helwig Department of Biological and Agricultural Engineering, Kansas State University, Manhattan, KS 66506, USA\\
\textsuperscript{6}Panhandle Research, Extension, and Education Center, University of Nebraska--Lincoln, Scottsbluff, NE 69361, USA\\
\textsuperscript{7}School of Computing, University of Nebraska--Lincoln, Lincoln, NE 68588, USA\\
\textsuperscript{8}Department of Agronomy and Horticulture, University of Nebraska--Lincoln, Lincoln, NE 68583, USA\\
\textsuperscript{9}Texas A\&M AgriLife Research and Extension Center, Texas A\&M University, 1102 East Drew Street, Lubbock, TX 79403, USA\\
\textsuperscript{10}Department of Agricultural and Biosystems Engineering, Makerere University, Kampala P.O. Box 7062, Uganda

\vspace{1em}
\begin{abstract}
Your abstract text goes here.
\end{abstract}

\vspace{1em}
\noindent
\textsuperscript{*}Corresponding author: \href{mailto:abia.katimbo@unl.edu}{abia.katimbo@unl.edu} (A. Katimbo)

\section{Introduction}

\noindent
Introduction text goes here. Add citations as needed \cite{reference1}.

\section{Methods}

\subsection{Study Site and Experimental Design}
\label{sec:study-site}

Description of study site and experimental design goes here.

\begin{itemize}
    \item Treatment 1: Brief description
    \item Treatment 2: Brief description
    \item Treatment 3: Brief description
    \item Treatment 4: Brief description
\end{itemize}

\begin{figure}[htbp]
    \centering
    % \includegraphics[width=\textwidth]{figures/figure1.png}
    \caption{Caption for your figure.}
    \label{fig:experimental-layout}
\end{figure}

\subsection{Data Acquisition and Sensor Setup}

Description of data acquisition and sensor setup goes here.

\subsubsection{Sensor Installation Protocols}

Sensor installation details go here.

\subsubsection{Data Logger Configuration and Hardware Integration}

Data logger configuration details go here.

\begin{table}[ht]
  \centering
  \caption{Description of Sensor Mapping}
  \label{tab:sensor-mapping}
  \begin{tabular}{|l|l|l|}
  \hline
  \textbf{Attribute} & \textbf{Sensor 1} & \textbf{Sensor 2} \\ \hline
  Example 1 & Value & Value \\ \hline
  Example 2 & Value & Value \\ \hline
  \end{tabular}
\end{table}

\begin{figure}[htbp]
    \centering
    % \includegraphics[width=\textwidth]{figures/figure2.png}
    \caption{Caption for your figure.}
    \label{fig:sensor-mapping}
\end{figure}

\subsection{Computation of Water Stress Indices and Crop Evapotranspiration}

Description of computation methods goes here.

\begin{figure}[htbp]
    \centering
    % \includegraphics[width=\textwidth]{figures/figure3.png}
    \caption{Caption for your figure.}
    \label{fig:cloud-functions}
\end{figure}

\subsubsection{Index One}

Description of first index calculation.

\begin{equation}
\text{Index} = \frac{(X - Y) - (X - Y)_L}{(X - Y)_U - (X - Y)_L}
\label{eq:index1}
\end{equation}

\subsubsection{Index Two}

Description of second index calculation.

\begin{equation}
\text{SWSI} = \frac{A - B}{C - D}
\label{eq:index2}
\end{equation}

\subsubsection{Reference Calculation}

Description of reference calculation.

\begin{equation}
\text{Reference} = \frac{0.408\Delta (R_n - G) + \gamma \frac{900}{T + 273} u_2 \text{VPD}}{\Delta + \gamma (1 + 0.34 u_2)}
\label{eq:reference}
\end{equation}

\subsection{Data Communication and Processing Infrastructure}

Description of data communication and processing infrastructure goes here.

\begin{figure}[htbp]
    \centering
    % \includegraphics[width=\textwidth]{figures/figure4.png}
    \caption{Caption for your figure.}
    \label{fig:data-flow}
\end{figure}

\subsection{Scheduling Methods}

Description of scheduling methods goes here.

\subsubsection{Method One}

Description of first method.

\begin{equation}
\text{Decision} = \begin{cases}
\text{Option A} & \text{if Condition A} \\
\text{Option B} & \text{if Condition B} \\
\text{No decision} & \text{if Condition C}
\end{cases}
\label{eq:decision_rule1}
\end{equation}

\subsubsection{Method Two}

Description of second method.

\begin{equation}
\text{Decision} = \begin{cases}
\text{Option A} & \text{if Condition A} \\
\text{Option B} & \text{if Condition B} \\
\text{No decision} & \text{if Condition C}
\end{cases}
\label{eq:decision_rule2}
\end{equation}

\subsubsection{Method Three}

Description of third method.

\begin{equation}
\text{Value} = A \times B + C \times D
\label{eq:composite}
\end{equation}

\subsubsection{Fuzzy Logic-Based Method}

Description of fuzzy logic method.

\begin{figure}[h!]
\centering
% \includegraphics[width=0.9\textwidth]{figures/figure5.png}
\caption{Caption for your figure.}
\label{fig:membership}
\end{figure}

\begin{figure}[h!]
\centering
% \includegraphics[width=0.9\textwidth]{figures/figure6.png}
\caption{Caption for your figure.}
\label{fig:rule_strength}
\end{figure}

\begin{figure}[h!]
\centering
% \includegraphics[width=0.9\textwidth]{figures/figure7.png}
\caption{Caption for your figure.}
\label{fig:defuzzification}
\end{figure}

\subsection{Data Quality, Reliability, and Security}

Description of data quality, reliability, and security measures goes here.

\begin{itemize}
    \item \textbf{Item 1:} Description
    \item \textbf{Item 2:} Description
    \item \textbf{Item 3:} Description
    \item \textbf{Item 4:} Description
\end{itemize}

\section{Results and Discussion}

\subsection{Main Dashboard}

Description of main dashboard goes here.

\begin{figure}[htbp]
    \centering
    % \includegraphics[width=\textwidth]{figures/figure8.png}
    \caption{Caption for your figure.}
    \label{fig:dashboard}
\end{figure}

\subsection{Visualization and Recommendations}

Description of visualizations and recommendations goes here.

\begin{figure}[htbp]
    \centering
    % \includegraphics[width=\textwidth]{figures/figure9.png}
    \caption{Caption for your figure.}
    \label{fig:dashboard-sample}
\end{figure}

\subsection{System Performance and Data Reliability}

\subsubsection{Packet Delivery Analysis}

Description of packet delivery analysis goes here.

\begin{figure}[htbp]
    \centering
    % \includegraphics[width=\textwidth]{figures/figure10.png}
    \caption{Caption for your figure.}
    \label{fig:packet-analysis}
\end{figure}

\subsubsection{System Power Consumption}

Description of system power consumption goes here.

\begin{figure}[htbp]
    \centering
    % \includegraphics[width=\textwidth]{figures/figure11.png}
    \caption{Caption for your figure.}
    \label{fig:battery-voltage}
\end{figure}

\subsection{Challenges and Future Directions}

\subsubsection{Challenges}

Description of challenges goes here.

\subsubsection{Future Directions and Research}

Description of future directions and research goes here.

\subsubsection{Adoption Considerations}

Description of adoption considerations goes here.

\section{Conclusion}

Conclusion text goes here.

%%%%%%%%%%%%%%%%%%%%%%%%%%%%%%%%%%%%%%%%%%%%%%%%%%%%%%%%%%%%
%             References & Appendices
%%%%%%%%%%%%%%%%%%%%%%%%%%%%%%%%%%%%%%%%%%%%%%%%%%%%%%%%%%%%
\clearpage
\printbibliography[heading=bibintoc, title={References}]
\clearpage
\appendix

\section{Supplemental Information}

% -----------------------------------------------------------------------
% Example of a portrait-oriented table
% -----------------------------------------------------------------------
\begin{table}[htbp]
    \centering
    \caption{Portrait Orientation Table Example}
    \begin{tabular}{lcc}
    \toprule
    \textbf{Parameter} & \textbf{Value 1} & \textbf{Value 2} \\
    \midrule
    Parameter A & 12.4 & 13.6 \\
    Parameter B & 7.3 & 8.1 \\
    Parameter C & 5.8 & 6.2 \\
    \bottomrule
    \end{tabular}
    \label{tab:portrait-example}
\end{table}

% -----------------------------------------------------------------------
% Example of a landscape-oriented table using sidewaystable
% -----------------------------------------------------------------------
\begin{sidewaystable}[htbp]
    \centering
    \caption{Landscape Orientation Table Example}
    \begin{tabular}{lcccccc}
    \toprule
    \textbf{Parameter} & \textbf{Jan} & \textbf{Feb} & \textbf{Mar} & \textbf{Apr} & \textbf{May} & \textbf{Jun} \\
    \midrule
    Temperature (°C) & 4.2 & 5.6 & 10.1 & 15.5 & 20.2 & 24.1 \\
    Rainfall (mm)    & 52  & 41  & 67   & 72   & 81   & 95   \\
    Evaporation (mm) & 25  & 30  & 45   & 60   & 70   & 85   \\
    \bottomrule
    \end{tabular}
    \label{tab:landscape-example}
\end{sidewaystable}

\clearpage

\section*{Summary of Best Practices}
\begin{itemize}
    \item \textbf{Author Formatting:} Let author names wrap naturally in the \texttt{\textbackslash author} field so LaTeX can manage line breaks.
    \item \textbf{Tables:} Use the regular \texttt{table} environment for portrait tables and \texttt{sidewaystable} (from the \texttt{rotating} package) for wide or landscape-oriented tables.
    \item \textbf{Appendices:} Consider placing large supplemental tables and figures in the appendix for clarity and organization.
\end{itemize}

\end{document}
