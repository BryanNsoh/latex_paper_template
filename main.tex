\documentclass[12pt]{article}

% Encoding and language
\usepackage[utf8]{inputenc}
\usepackage[T1]{fontenc}
\usepackage[english]{babel}

% Page layout
\usepackage[margin=1in]{geometry}

% Math and symbols
\usepackage{amsmath,amssymb,amsfonts}

% For quotes & punctuation with biblatex + babel
\usepackage{csquotes}

\usepackage[
  backend=biber,
  style=numeric,
  sorting=none,
  giveninits=true,
  doi=true,
  url=false,
  isbn=false,
  eprint=false,
  maxbibnames=10,    % Elsevier typically uses 10 authors before et al.
  minbibnames=10,    % Ensures at least 10 authors appear before et al.
  maxcitenames=2,
  uniquename=false
]{biblatex}
\addbibresource{references.bib}

% Author initials first (Z. Gu)
\DeclareNameAlias{author}{given-family}

% No "In:" prefix
\renewbibmacro{in:}{}

% Title without quotation marks or italics
\DeclareFieldFormat[article,inproceedings,incollection]{title}{#1}

% Pages format without prefix
\DeclareFieldFormat[article]{pages}{#1}

% DOI without prefix
\DeclareFieldFormat{doi}{#1}

% Hyperlinks
\usepackage{hyperref}
\hypersetup{
  colorlinks  = true,
  linkcolor   = blue,
  citecolor   = blue,
  urlcolor    = magenta
}
\usepackage{url}

% Graphics, tables, and floats
\usepackage{graphicx}
\usepackage{booktabs}
\usepackage{xltabular}
\usepackage{float}

% For referencing figures/tables with bold labels
\usepackage[labelfont=bf]{caption}

% Optional line spacing
\usepackage{setspace}
%\onehalfspacing

% -------------------------------
% Code Listings Configuration
% -------------------------------
\usepackage{listings}
\usepackage{xcolor} % for custom colors

% Define a custom style for Python code
\lstdefinestyle{myCustomPython}{
  language=Python,
  basicstyle=\ttfamily\footnotesize,
  keywordstyle=\color{blue}\bfseries,
  commentstyle=\color{green!50!black}\itshape,
  stringstyle=\color{red},
  backgroundcolor=\color{gray!10},
  frame=single,
  rulecolor=\color{gray},
  numbers=left,
  numberstyle=\tiny\color{gray},
  stepnumber=1,
  breaklines=true,
  breakatwhitespace=true,
  showstringspaces=false,
  captionpos=b,
  tabsize=4,
  columns=fullflexible
}

% Set the default style to your custom style
\lstset{style=myCustomPython}
% -------------------------------

%%%%%%%%%%%%%%%%%%%%%%%%%%%%%%%%%%%%%%%%%%%%%%%%%%%%%%%%%%%%
%              BEGIN DOCUMENT
%%%%%%%%%%%%%%%%%%%%%%%%%%%%%%%%%%%%%%%%%%%%%%%%%%%%%%%%%%%%
\begin{document}

\title{Your Paper Title Here}
\author{%
  Author Name\textsuperscript{1,2},
  Second Author\textsuperscript{1,2}\thanks{\textit{Corresponding author:} \texttt{email@domain.com}}~,
  et al.
}
\date{\vspace{-1em}} % optional
\maketitle

\noindent
\textsuperscript{1}Department Name, University Name, City, State, Country

\noindent
\textsuperscript{2}Center Name, University Name, City, State, Country

\vspace{1em}
\begin{abstract}
Your abstract text goes here.
\end{abstract}

\section{Introduction}

\noindent
Introduction text goes here. Add citations as needed \cite{reference1}.

\section{Methods}

\subsection{Study Site and Experimental Design}
\label{sec:study-site}

Description of study site and experimental design goes here.

\begin{itemize}
    \item Treatment 1: Brief description
    \item Treatment 2: Brief description
    \item Treatment 3: Brief description
    \item Treatment 4: Brief description
\end{itemize}

\begin{figure}[htbp]
    \centering
    % \includegraphics[width=\textwidth]{figures/figure1.png}
    \caption{Caption for your figure.}
    \label{fig:experimental-layout}
\end{figure}

\subsection{Data Acquisition and Sensor Setup}

Description of data acquisition and sensor setup goes here.

\subsubsection{Sensor Installation Protocols}

Sensor installation details go here.

\subsubsection{Data Logger Configuration and Hardware Integration}

Data logger configuration details go here.

\begin{table}[ht]
  \centering
  \caption{Description of Sensor Mapping}
  \label{tab:sensor-mapping}
  \begin{tabular}{|l|l|l|}
  \hline
  \textbf{Attribute} & \textbf{Sensor 1} & \textbf{Sensor 2} \\ \hline
  Example 1 & Value & Value \\ \hline
  Example 2 & Value & Value \\ \hline
  \end{tabular}
\end{table}

\begin{figure}[htbp]
    \centering
    % \includegraphics[width=\textwidth]{figures/figure2.png}
    \caption{Caption for your figure.}
    \label{fig:sensor-mapping}
\end{figure}

\subsection{Computation of Water Stress Indices and Crop Evapotranspiration}

Description of computation methods goes here.

\begin{figure}[htbp]
    \centering
    % \includegraphics[width=\textwidth]{figures/figure3.png}
    \caption{Caption for your figure.}
    \label{fig:cloud-functions}
\end{figure}

\subsubsection{Index One}

Description of first index calculation.

\begin{equation}
\text{Index} = \frac{(X - Y) - (X - Y)_L}{(X - Y)_U - (X - Y)_L}
\label{eq:index1}
\end{equation}

Where $(X - Y)_L$ and $(X - Y)_U$ are the lower and upper bounds.

\subsubsection{Index Two}

Description of second index calculation.

\begin{equation}
\text{SWSI} = \frac{A - B}{C - D}
\label{eq:index2}
\end{equation}

\subsubsection{Reference Calculation}

Description of reference calculation.

\begin{equation}
\text{Reference} = \frac{0.408\Delta (R_n - G) + \gamma \frac{900}{T + 273} u_2 \text{VPD}}{\Delta + \gamma (1 + 0.34 u_2)}
\label{eq:reference}
\end{equation}

\subsection{Data Communication and Processing Infrastructure}

Description of data communication and processing infrastructure goes here.

\begin{figure}[htbp]
    \centering
    % \includegraphics[width=\textwidth]{figures/figure4.png}
    \caption{Caption for your figure.}
    \label{fig:data-flow}
\end{figure}

\subsection{Scheduling Methods}

Description of scheduling methods goes here.

\subsubsection{Method One}

Description of first method.

\begin{equation}
\text{Decision} = \begin{cases}
\text{Option A} & \text{if Condition A} \\
\text{Option B} & \text{if Condition B} \\
\text{No decision} & \text{if Condition C}
\end{cases}
\label{eq:decision_rule1}
\end{equation}

\subsubsection{Method Two}

Description of second method.

\begin{equation}
\text{Decision} = \begin{cases}
\text{Option A} & \text{if Condition A} \\
\text{Option B} & \text{if Condition B} \\
\text{No decision} & \text{if Condition C}
\end{cases}
\label{eq:decision_rule2}
\end{equation}

\subsubsection{Method Three}

Description of third method.

\begin{equation}
\text{Value} = A \times B + C \times D
\label{eq:composite}
\end{equation}

\subsubsection{Fuzzy Logic-Based Method}

Description of fuzzy logic method.

\begin{figure}[h!]
\centering
% \includegraphics[width=0.9\textwidth]{figures/figure5.png}
\caption{Caption for your figure.}
\label{fig:membership}
\end{figure}

\begin{figure}[h!]
\centering
% \includegraphics[width=0.9\textwidth]{figures/figure6.png}
\caption{Caption for your figure.}
\label{fig:rule_strength}
\end{figure}

\begin{figure}[h!]
\centering
% \includegraphics[width=0.9\textwidth]{figures/figure7.png}
\caption{Caption for your figure.}
\label{fig:defuzzification}
\end{figure}

\subsection{Data Quality, Reliability, and Security}

Description of data quality, reliability, and security measures goes here.

\begin{itemize}
    \item \textbf{Item 1:} Description
    \item \textbf{Item 2:} Description
    \item \textbf{Item 3:} Description
    \item \textbf{Item 4:} Description
\end{itemize}

\section{Results and Discussion}

\subsection{Main Dashboard}

Description of main dashboard goes here.

\begin{figure}[htbp]
    \centering
    % \includegraphics[width=\textwidth]{figures/figure8.png}
    \caption{Caption for your figure.}
    \label{fig:dashboard}
\end{figure}

\subsection{Visualization and Recommendations}

Description of visualizations and recommendations goes here.

\begin{figure}[htbp]
    \centering
    % \includegraphics[width=\textwidth]{figures/figure9.png}
    \caption{Caption for your figure.}
    \label{fig:dashboard-sample}
\end{figure}

\subsection{System Performance and Data Reliability}

\subsubsection{Packet Delivery Analysis}

Description of packet delivery analysis goes here.

\begin{figure}[htbp]
    \centering
    % \includegraphics[width=\textwidth]{figures/figure10.png}
    \caption{Caption for your figure.}
    \label{fig:packet-analysis}
\end{figure}

\subsubsection{System Power Consumption}

Description of system power consumption goes here.

\begin{figure}[htbp]
    \centering
    % \includegraphics[width=\textwidth]{figures/figure11.png}
    \caption{Caption for your figure.}
    \label{fig:battery-voltage}
\end{figure}

\subsection{Challenges and Future Directions}

\subsubsection{Challenges}

Description of challenges goes here.

\subsubsection{Future Directions and Research}

Description of future directions and research goes here.

\subsubsection{Adoption Considerations}

Description of adoption considerations goes here.

\section{Conclusion}

Conclusion text goes here.

%%%%%%%%%%%%%%%%%%%%%%%%%%%%%%%%%%%%%%%%%%%%%%%%%%%%%%%%%%%%
%             References & Appendices
%%%%%%%%%%%%%%%%%%%%%%%%%%%%%%%%%%%%%%%%%%%%%%%%%%%%%%%%%%%%
\clearpage
\printbibliography[heading=bibintoc, title={References}]
\clearpage
\appendix
\section{Appendix A: Cloud Functions}
\label{app:cloud-functions}

This appendix provides information about the cloud functions used in the data processing pipeline.

\begin{lstlisting}
def weather_processing_function(event, context):
    """
    Cloud function to process weather data and compute metrics.
    
    Parameters:
    -----------
    event: dict
        The event payload
    context: google.cloud.functions.Context
        The Cloud Functions event metadata
    """
    # Retrieve weather data from external API
    weather_data = fetch_weather_data(API_KEY, location)
    
    # Compute reference evapotranspiration
    eto = compute_reference_evapotranspiration(
        weather_data['temperature'], 
        weather_data['humidity'], 
        weather_data['solar_radiation'], 
        weather_data['wind_speed'])
    
    # Store results in database
    store_results_in_database(eto, weather_data)
    
    return None
\end{lstlisting}

\begin{figure}[h!]
\centering
% \includegraphics[width=0.8\textwidth]{figures/cloud_functions_diagram.png}
\caption{Diagram showing interactions between cloud functions in the data processing pipeline.}
\label{fig:cloud-functions-diagram}
\end{figure}

\section{Appendix B: Virtual Machine MQTT Client}
\label{app:vm-mqtt}

This appendix contains sample code used in the virtual machine to maintain an active MQTT client connection.

\begin{lstlisting}
import paho.mqtt.client as mqtt
import json
from google.cloud import bigquery

# Configuration
MQTT_BROKER = "mqtt.example.com"
MQTT_PORT = 1883
MQTT_TOPIC = "sensors/data/#"

# Callbacks
def on_connect(client, userdata, flags, rc):
    print(f"Connected with result code {rc}")
    client.subscribe(MQTT_TOPIC)

def on_message(client, userdata, msg):
    try:
        payload = msg.payload.decode("utf-8")
        data = json.loads(payload)
        processed_data = process_data(data)
        insert_into_bigquery(processed_data)
    except Exception as e:
        print(f"Error processing message: {e}")

# Set up client
client = mqtt.Client()
client.on_connect = on_connect
client.on_message = on_message
client.connect(MQTT_BROKER, MQTT_PORT, 60)

# Start loop
try:
    client.loop_forever()
except KeyboardInterrupt:
    print("Stopped by user")
finally:
    client.disconnect()
\end{lstlisting}

\section{Appendix C: Fuzzy Logic Rules}
\label{app:fuzzy-rules}

This appendix details the fuzzy logic rules used in the decision-making process.

\begin{table}[ht]
  \centering
  \caption{Fuzzy Logic Rule Set}
  \label{tab:fuzzy-rules}
  \begin{tabular}{|l|l|l|l|l|p{5cm}|}
  \hline
  \textbf{Rule} & \textbf{Input 1} & \textbf{Input 2} & \textbf{Input 3} & \textbf{Output} & \textbf{Description} \\ \hline
  Rule 1 & CWSI is very\_low & SWSI is very\_low & - & Irrigation is none & When both indices indicate no stress \\ \hline
  Rule 2 & CWSI is high & SWSI is medium & - & Irrigation is medium & Moderate response to stress \\ \hline
  Rule 3 & CWSI is very\_high & SWSI is high & - & Irrigation is high & Strong response to severe stress \\ \hline
  Rule 4 & CWSI is medium & SWSI is medium & ETo is medium & Irrigation is medium & Balanced response \\ \hline
  Rule 5 & CWSI is low & SWSI is high & ETo is high & Irrigation is medium & Prioritize soil conditions \\ \hline
  \end{tabular}
\end{table}

\begin{lstlisting}
def apply_fuzzy_rule(cwsi, swsi, eto):
    """
    Apply fuzzy logic rule to determine irrigation amount
    
    Parameters:
    -----------
    cwsi: float
        Crop Water Stress Index value
    swsi: float
        Soil Water Stress Index value
    eto: float
        Reference Evapotranspiration value
        
    Returns:
    --------
    float
        Recommended irrigation amount
    """
    # Example rule implementation
    if cwsi < 0.2 and swsi < 0.2:
        return 0.0  # No irrigation
    elif cwsi > 0.7 and swsi > 0.5:
        return 25.4  # Full irrigation
    else:
        # Apply fuzzy logic to determine amount
        strength = min(
            membership_medium(cwsi),
            membership_medium(swsi),
            membership_medium(eto)
        )
        return defuzzify(strength)
\end{lstlisting}

\section{Appendix D: Statistical Analysis}
\label{app:statistical}

This appendix contains example statistical analysis code used to process experimental data.

\begin{lstlisting}
import pandas as pd
import numpy as np
from scipy import stats

def analyze_treatment_effects(data_file):
    """
    Analyze treatment effects on water use efficiency
    
    Parameters:
    -----------
    data_file: str
        Path to CSV file with experimental data
    """
    # Load data
    df = pd.read_csv(data_file)
    
    # Group by treatment
    grouped = df.groupby('Treatment')
    
    # Calculate mean and standard deviation
    summary = grouped.agg({
        'WaterUse': ['mean', 'std'],
        'Yield': ['mean', 'std'],
        'WUE': ['mean', 'std']
    })
    
    # Perform ANOVA
    treatments = df['Treatment'].unique()
    anova_data = [df[df['Treatment'] == t]['WUE'] for t in treatments]
    f_stat, p_value = stats.f_oneway(*anova_data)
    
    print(f"ANOVA results: F={f_stat:.2f}, p={p_value:.4f}")
    return summary
\end{lstlisting}

\end{document}