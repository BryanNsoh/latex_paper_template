\section{Appendix A: Cloud Functions}
\label{app:cloud-functions}

This appendix provides information about the cloud functions used in the data processing pipeline.

\begin{lstlisting}
def weather_processing_function(event, context):
    """
    Cloud function to process weather data and compute metrics.
    
    Parameters:
    -----------
    event: dict
        The event payload
    context: google.cloud.functions.Context
        The Cloud Functions event metadata
    """
    # Retrieve weather data from external API
    weather_data = fetch_weather_data(API_KEY, location)
    
    # Compute reference evapotranspiration
    eto = compute_reference_evapotranspiration(
        weather_data['temperature'], 
        weather_data['humidity'], 
        weather_data['solar_radiation'], 
        weather_data['wind_speed'])
    
    # Store results in database
    store_results_in_database(eto, weather_data)
    
    return None
\end{lstlisting}

\begin{figure}[h!]
\centering
% \includegraphics[width=0.8\textwidth]{figures/cloud_functions_diagram.png}
\caption{Diagram showing interactions between cloud functions in the data processing pipeline.}
\label{fig:cloud-functions-diagram}
\end{figure}

\section{Appendix B: Virtual Machine MQTT Client}
\label{app:vm-mqtt}

This appendix contains sample code used in the virtual machine to maintain an active MQTT client connection.

\begin{lstlisting}
import paho.mqtt.client as mqtt
import json
from google.cloud import bigquery

# Configuration
MQTT_BROKER = "mqtt.example.com"
MQTT_PORT = 1883
MQTT_TOPIC = "sensors/data/#"

# Callbacks
def on_connect(client, userdata, flags, rc):
    print(f"Connected with result code {rc}")
    client.subscribe(MQTT_TOPIC)

def on_message(client, userdata, msg):
    try:
        payload = msg.payload.decode("utf-8")
        data = json.loads(payload)
        processed_data = process_data(data)
        insert_into_bigquery(processed_data)
    except Exception as e:
        print(f"Error processing message: {e}")

# Set up client
client = mqtt.Client()
client.on_connect = on_connect
client.on_message = on_message
client.connect(MQTT_BROKER, MQTT_PORT, 60)

# Start loop
try:
    client.loop_forever()
except KeyboardInterrupt:
    print("Stopped by user")
finally:
    client.disconnect()
\end{lstlisting}

\section{Appendix C: Fuzzy Logic Rules}
\label{app:fuzzy-rules}

This appendix details the fuzzy logic rules used in the decision-making process.

\begin{table}[ht]
  \centering
  \caption{Fuzzy Logic Rule Set}
  \label{tab:fuzzy-rules}
  \begin{tabular}{|l|l|l|l|l|p{5cm}|}
  \hline
  \textbf{Rule} & \textbf{Input 1} & \textbf{Input 2} & \textbf{Input 3} & \textbf{Output} & \textbf{Description} \\ \hline
  Rule 1 & CWSI is very\_low & SWSI is very\_low & - & Irrigation is none & When both indices indicate no stress \\ \hline
  Rule 2 & CWSI is high & SWSI is medium & - & Irrigation is medium & Moderate response to stress \\ \hline
  Rule 3 & CWSI is very\_high & SWSI is high & - & Irrigation is high & Strong response to severe stress \\ \hline
  Rule 4 & CWSI is medium & SWSI is medium & ETo is medium & Irrigation is medium & Balanced response \\ \hline
  Rule 5 & CWSI is low & SWSI is high & ETo is high & Irrigation is medium & Prioritize soil conditions \\ \hline
  \end{tabular}
\end{table}

\begin{lstlisting}
def apply_fuzzy_rule(cwsi, swsi, eto):
    """
    Apply fuzzy logic rule to determine irrigation amount
    
    Parameters:
    -----------
    cwsi: float
        Crop Water Stress Index value
    swsi: float
        Soil Water Stress Index value
    eto: float
        Reference Evapotranspiration value
        
    Returns:
    --------
    float
        Recommended irrigation amount
    """
    # Example rule implementation
    if cwsi < 0.2 and swsi < 0.2:
        return 0.0  # No irrigation
    elif cwsi > 0.7 and swsi > 0.5:
        return 25.4  # Full irrigation
    else:
        # Apply fuzzy logic to determine amount
        strength = min(
            membership_medium(cwsi),
            membership_medium(swsi),
            membership_medium(eto)
        )
        return defuzzify(strength)
\end{lstlisting}

\section{Appendix D: Statistical Analysis}
\label{app:statistical}

This appendix contains example statistical analysis code used to process experimental data.

\begin{lstlisting}
import pandas as pd
import numpy as np
from scipy import stats

def analyze_treatment_effects(data_file):
    """
    Analyze treatment effects on water use efficiency
    
    Parameters:
    -----------
    data_file: str
        Path to CSV file with experimental data
    """
    # Load data
    df = pd.read_csv(data_file)
    
    # Group by treatment
    grouped = df.groupby('Treatment')
    
    # Calculate mean and standard deviation
    summary = grouped.agg({
        'WaterUse': ['mean', 'std'],
        'Yield': ['mean', 'std'],
        'WUE': ['mean', 'std']
    })
    
    # Perform ANOVA
    treatments = df['Treatment'].unique()
    anova_data = [df[df['Treatment'] == t]['WUE'] for t in treatments]
    f_stat, p_value = stats.f_oneway(*anova_data)
    
    print(f"ANOVA results: F={f_stat:.2f}, p={p_value:.4f}")
    return summary
\end{lstlisting}